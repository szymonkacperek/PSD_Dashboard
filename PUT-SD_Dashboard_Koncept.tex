%% PODSTAWOWE USTAWIENIA DOKUMENTU
\documentclass[12pt, a4paper, polish]{article}
%\usepackage[a4paper, lmargin=2.5cm, rmargin=2.5cm, hmargin=2.5cm, bmargin=2.5cm]{geometry}
\usepackage[a4paper,top=1cm,bottom=0.5cm,left=0.5cm,right=0.5cm]{geometry}
%\geometry{verbose,lmargin=2.5cm,rmargin=2.5cm}
\usepackage{enumerate}
% Symbole matematyczne
\usepackage{latexsym}
% Formatowanie czcionki
\usepackage[T1]{fontenc}
% Formatowanie polskich znaków
\usepackage{polski}
\usepackage[utf8]{inputenc}
% Akapit po sekcji
%\usepackage{indentfirst}
% Ustawienie nagłówków i stopek
\usepackage{fancyhdr}
% Liczba stron
\usepackage{lastpage}
% Kolumny
\usepackage{paracol}
% Czcionka latin modern
\usepackage{lmodern}
% Równania matematyczne
\usepackage{amsmath}
\usepackage{amsfonts}
\usepackage{amssymb}
\usepackage{amsthm}
% Wstawianie grafik
\usepackage{graphicx}
%\usepackage[section]{placeins} % Ogarnięcie obrazków
\usepackage{float}
\usepackage{subcaption}
%\usepackage[outdir=./]{epstopdf}
\usepackage{grffile}

%% ODSTĘPY W WIERSZACH
\setlength{\parindent}{1.5cm}
\setlength{\parskip}{0.5cm}
\linespread{1}

% SZARE TŁO TEKSTU
\usepackage[most]{tcolorbox}
\tcbset{
	frame code={}
	center title,
	left=0pt,
	right=0pt,
	top=0pt,
	bottom=0pt,
	colback=gray!30,
	colframe=white,
	width=\dimexpr\textwidth\relax,
	enlarge left by=0mm,
	boxsep=5pt,
	arc=0pt,outer arc=0pt,
}

% HIPERŁĄCZA SPISU TREŚCI
\usepackage{hyperref}
\hypersetup{
	colorlinks,
	citecolor=black,
	filecolor=black,
	linkcolor=black,
	urlcolor=black
}

\begin{document}
	\fancyhf{}	% Usunięcie domyślnego stylu numerowania
	
	% ********************** TYTUŁ ****************************
	\noindent\textsf{\begin{Large}PUT Solar Dynamics\\\end{Large}
		Koncept \textbf{Dashboard}\\
		Szymon Kacperek\\
		\rule{\columnwidth}{0.2pt}}
	
	\thispagestyle{empty}
	\pagestyle{fancy}
	\fancyhead{}
	\rhead{\thepage}
	\renewcommand{\headrulewidth}{0pt}%{}
	\setlength{\footskip}{1mm}
	
	%							*****************POCZĄTEK DOKUMENTU*****************
\section{Zasada działania}
Do ekranów OLED wykorzystana jest biblioteka u8glib (wg tutoriala http://elastic-notes.blogspot.com/p/stm32-i2c-oled-ssd.html). 

LEDami odpowiadającymi za wizualizację prędkości zajmuje się procesor F030. Komunikuje się on z F103 za pomocą UART (piny PA10 i PA9, oznaczone jako MCU2\_RX/TX), a następnie F030 za pomocą SPI steruje LEDami prędkości. Program na F030 został wzięty z elektrody i zmodowany. Diody te mają unikalny adres i mogą być sterowane z jednego pina mikrokontrolera.

Interfejs I2C użyty jest w mniejszych OLED (Display I), które są multipleksowane i sygnał jest później rozdzielany.




\end{document}
